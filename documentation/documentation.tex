\documentclass[10pt]{article}
\usepackage[pdftex]{graphicx}
\usepackage{amssymb, amsmath}

\usepackage{fullpage}
\setlength{\parindent}{0pt}
\setlength{\parskip}{\baselineskip}

\begin{document} 

\section{Equations}

If we define

\begin{equation}
    P(V,t) \equiv P(V(t) \cap  V(s) < V_{th} \forall s < t)
\end{equation}

Informally This refers to the probability of $V(t)$, the membrane potential,
being less than $V_{th}$ the firing threshold until time $t$.  


We need to solve the following fokker-planck drift diffusion equation
numerically.

\begin{equation}
    \frac{\partial P(V,t)}{\partial t} =
    \frac{1}{2} \frac{\partial^2 P(V,t) } {\partial V^2} +
    g\frac{\partial[(V-V_{rest})]P(V,t)}{\partial V}
\end{equation}

Given the boundary conditions:

\begin{equation}
    P(V_{th},t) = 0
\end{equation}

\begin{equation}
    P(V,0) = \delta(V-V_{reset})
\end{equation}

And the definition of V-rest.


We can rewrite this as:

\begin{equation}
    \frac{1}{2} \frac{\partial^2 P(V,t) } {\partial V^2} +
    g(V-V_{rest})\frac{\partial P(V,t)}{\partial V} +
    gP(V,t) -
    \frac{\partial P(V,t)}{\partial t} = 
    0
\end{equation}

Since we know that $ V-V_{rest} = 0 $

Next we discretize time and potential. Time is discretized  into $U$
intervals of length $u$ and indexed by: $ 0,1, \dots \tau $ Potential
is discretized into $W$ intervals of length $w$ and indexed by $ 0,1,
\dots \nu $. We adopt the notation that $P_{\nu,\tau} = P(\nu,\tau)$ 

Using the Crank-Nicolson scheme we  may now rewrite the derivatives as:

\begin{equation}
    P = \frac{P_{\nu,\tau} + P_{\nu,\tau + 1}}{2}
\end{equation}

\begin{equation}
    \frac{\partial P}{\partial t} = \frac{P_{\nu,\tau +1 } -
    P_{\nu,\tau}}{u}
\end{equation}

\begin{equation}
    \frac{\partial P}{\partial V} = 
    \frac{P_{\nu +1,\tau } + P_{\nu +1,\tau +1 } -
    P_{\nu - 1,\tau } P_{\nu -1,\tau +1}}
    {4w}
\end{equation}

\begin{equation}
    \frac{\partial^2 P}{\partial V^2} = 
    \frac{P_{\nu+1,\tau} - 2 P_{\nu,\tau} + P_{\nu-1,\tau} +
    P_{\nu+1,\tau+1} - 2 P_{\nu,\tau+1} + P_{\nu-1,\tau+1}}
    {2w^2}
\end{equation}

if we now let:

\begin{align*}
a &= \frac{1}{2} \\
b &= g(V - V_{rest}) \\
c &= g \\
\end{align*}

and multiply throughout with $4w^2u$

we may rearrange all the $P_{*,\tau+1} $ terms on the left hand side

\begin{multline}
    \overbrace{-(2au+bwu)} P_{\nu+1,\tau+1} + 
    \overbrace{(4au - 2cw^2u + 4w^2)} P_{\nu,\tau+1} 
    \overbrace{-(2au-bwu)} P_{\nu-1,\tau+1} 
    =  \\
    (2au+bwu) P_{\nu+1,\tau} +  
    (-4au +i  2cw^2u + 4w^2) P_{\nu,\tau} + 
    (2au-bwu) P_{\nu-1,\tau} 
\end{multline}

We note here that we have obtained $W-1$ simultaneous equations which
we may now rewrite in the following tridiagonal matrix notation.



\end{document}
